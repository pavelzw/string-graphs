\documentclass[12pt]{article}

\usepackage[margin=2cm]{geometry}

% BibTeX setup
% \usepackage[backend=bibtex, bibstyle=alphabetic, citestyle=alphabetic]{biblatex}
% \bibliography{references}

\usepackage[english]{babel} % English

% Standard packages for math-related things.
\usepackage{amsmath}
\usepackage{amssymb}
\usepackage{amsthm}
\usepackage{mathtools}

\usepackage{stmaryrd} % for \lightning

\usepackage{graphicx} % to include graphics with \includegraphics[options]{imagefile}

\theoremstyle{plain} % Usual style for theorems, etc.
\newtheorem{theorem}{Theorem}
\newtheorem{lemma}[theorem]{Lemma}
\newtheorem{corollary}[theorem]{Corollary}

\newcommand{\N}{\mathbb{N}} % blackboard bold N for natural numbers
\newcommand{\R}{\mathbb{R}} % blackboard bold R for real numbers
\newcommand{\Z}{\mathbb{Z}} % blackboard bold Z for integers

\newcommand{\set}[1]{\{#1\}}
\newcommand{\image}[1]{\begin{center}\includegraphics[width=0.5\textwidth]{#1}\end{center}}

\begin{document}

\textbf{Example}
\image{images/figure-0.png}

Start with \(K_5\), create a new vertex per edge.
This cannot be drawn as a string graph. 
Assume we would be able to do it, then the curves should look 
somewhat like this (right). Contradiction to \(K_5\) having no planar drawing.

\begin{lemma}\label{lem:finite-intersections}
A string graph can be realized by a family of polygonal arcs with a finite number of intersections.
\end{lemma}

\begin{proof}
Assume string graph realized by \( (C_i)_{i\in I} \). Let \(C = \bigcup_{i\in I} C_i\).
\(C\) is compact since it is bounded and closed. Let \(p \in C\) and \(O \ni p\) an open neighborhood 
in the plane. \(O\) \textit{respects} the family of curves if \(O\) only intersects curves that contain \(p\).

\image{images/figure-1.jpeg}

Every point has at least one such neighborhood (curves are compact). 
Let \(\mathcal{O}\) be the colection of all open neighborhoods of \(p\in C\) that respect 
the family \((C_i)_{i\in I}\). 

\(\mathcal{O}\) is an open cover of \(C\). \(\Rightarrow\) it contains a finite subcover 
\(\mathcal{O}'\) of \(C\). \(\Rightarrow\) each curve \(C_i\) is contained 
in a finite collection \(\mathcal{O}_i' \subseteq \mathcal{O}'\) of open sets
none of whom contain any points of curves \(C_j\) that \(C_i\) does not intersect. 

Replace \(C_i\) with a simple polygonal arc \(P_i\) in \(\mathcal{O}_i'\) 
while guaranteeing that \(P_i\) intersects \(P_j\) if \(C_i\) and \(C_j\) intersect.
(Possible since \(\exists O \in \mathcal{O}'\) which contains a common intersection point of \(C_i\) and \(C_j\)).
\end{proof}

\begin{lemma}\ 
\begin{enumerate}
    \item \(c_w(m) \leq c_r(m)\)
    \item \(c_r(m) \leq d \cdot c_s(p(m))\) for some \(d > 0\), \(p\) polynomial
    \item \(c_s(m) \leq 4 c_w(2m) + 2m\)
\end{enumerate}
\end{lemma}

\begin{proof}
1. Since \(c_w(G, R) = c_r(G, R')\) for some \(R' \subset R\), it follows from \(c_w(G, R) \leq \max\set{c_r(G, R'): R' \subseteq R \text{ and } (G,R) \text{ has a realization}} = c_r(G, R') \).

2. Idea: Let \(G = (V, E)\) be a graph and \(R \subseteq \binom{E}{2}\). Construct \(V' = V \cup E \cup \set{(u, e) : u \in e \in E}\) 
and \(E' = R \cup \set{ \set{u, (u,e)}, u \in e \in E} \cup \set{\set{e, (u,e)}, u\in e \in E}\). Then \((G, R)\) is realizable iff 
\(G' = (V', E')\) is a string graph. \(\Rightarrow \) 2.

\image{images/figure-2.jpeg}

3. Idea: Given a graph \(G = (V, E)\), let \(G' = (V \cup E, \set{\set{u,e} : u \in e \in E})\) and 
\(R = \set{\set{\set{u,e}, \set{v,f}} : \set{u,v} \in E}\). Then \(G\) is a string graph iff \((G', R)\) 
is weakly realizable. \(\Rightarrow \) 3.

\image{images/figure-3.jpeg}
\end{proof}

\begin{lemma}\label{lem:even-occurences}
    Every word of length \(\geq 2^n\) over an alphabet of size \(n\) contains a non-trivial subword in which every character occurs an even number of times.
\end{lemma}

\begin{proof}
Let \(\Sigma = \set{1, \ldots, n}\), \(w \in \Sigma^*, |w| \geq 2^n\). To every \(i \in \set{0, \ldots, 2^n}\) 
assign a vector \(v_i \in \Z_2^n\) whose \(j\)-th coordinate \(=\) parity of \#occurences of symbol \(j\) in the 
prefix of \(w\) of length \(i\).

Example:
\(w = 12312231, |w| = 8 \geq 2^3\)

\[ v_0 = \begin{pmatrix}
0 \\ 0 \\ 0
\end{pmatrix}, v_1 = \begin{pmatrix}
1 \\ 0 \\ 0
\end{pmatrix}, v_2 = \begin{pmatrix}
1 \\ 1 \\ 0
\end{pmatrix}, v_3 = \begin{pmatrix}
1 \\ 1 \\ 1
\end{pmatrix}, v_4 = \begin{pmatrix}
0 \\ 1 \\ 1
\end{pmatrix}, \]

\[ v_5 = \begin{pmatrix}
0 \\ 0 \\ 1
\end{pmatrix}, v_6 = \begin{pmatrix}
0 \\ 1 \\ 1
\end{pmatrix}, v_7 = \begin{pmatrix}
0 \\ 1 \\ 0
\end{pmatrix}, v_8 = \begin{pmatrix}
1 \\ 1 \\ 0
\end{pmatrix} \]

\(\exists 2^n + 1\) indices \(\Rightarrow\) Pigeonhole Principle \(\Rightarrow\)
\(\exists i < j\) with \(v_i = v_j\). \(\Rightarrow\) \(w\) contains a subword \(w' \in \Sigma^*\)
starting in position \(i + 1\) and ending in position \(j\) such that every symbol occurs an even number of times.

Example: \(v_2 = v_8 \Rightarrow w' = w[2:] = 312231\).
\end{proof}

\begin{theorem}\label{thm:weak-realizability-bound}
    Let \(G\) be a graph with \(m\) edges, \(R \subseteq \binom{E}{2}\) such that \((G, R)\) is weakly realizable, and let \(D\) 
    be a weak realization of \((G, R)\) with the minimal number of intersections. Then for any edge \(e \in G\) 
    there are less than \(2^m\) intersections on the curve realizing \(e\) in \(D\).
\end{theorem}

\begin{proof}
Suppose not. Let \(D\) be a weak realization of \((G, R)\) with minimal \#intersections and \(e\) 
be an edge which has \(> 2^m - 1\) intersections. Lemma \ref{lem:finite-intersections} tells us that \# intersections is finite.
By Lemma \ref{lem:even-occurences}, we can choose a segment of \(e\) which is intersected an even number of times by any other edge.
Draw a window around this segment with no other intersections. 

\image{images/figure-4.pdf}

Let \(2n_f\) be the number of intersections of edge \(f\) with \(e\) in this window.
For each \(f\), assign numbers \(1, \ldots, 4 n_f\) to the intersections with the window in the 
order they appear on \(f\). 

\image{images/figure-5.pdf}

We can assume (by Jordan-Schoenflies theorem) that the window is a circle and \(e\) is a straight line passing through the center
and for every \(f\), the intersections \(2i-1\) and \(2i\) are mirror images of each other.

Next step: remove everything inside the window with the exception of \(e\). For each \(f\), there is a connection between 
\(4i-2\) and \(4i-1\) lying completely outside the window. 

\image{images/figure-6.pdf}

Use circular inversion along the circle to bring all of these conections inside.

\image{images/figure-7.pdf}

Now mirror everything inside of the window along \(e\).

\image{images/figure-8.pdf}

Now we have for every edge a connection between \(4i-3\) and \(4i\) which is inside the window.

Build a new version of \(f\): start at intersection \(1\) (connected to one of the endpoints of \(f\)),
continue to \(4\) (inside the window), etc. until \(4 n_f\).

\image{images/figure-9.pdf}

\(f\) intersects \(e\) an even number of times \(\Rightarrow \tilde{f}\) still connects the two original endpoints.

\(\leadsto\) reduced the number of intersections within the window from \(4 n_f\) 
to \(2 n_f\).

Every intersection between the curves inside the circle corresponds to an intersection 
outside \(\Rightarrow\) the new realization respects \(R\) (in this example, there are no intersections).

The number of intersections along \(e\) might have increased (a connection from outside the window brought inside might intersect \(e\) arbitrarily often).

We know: halved the number of intersections between the intersections and the boundary. 

\(\leadsto\) \(e'\) = one of the two sides of the boundary of \(e\)
(at least one side has less intersections than before).

\image{images/figure-10.pdf}

Overall, we have fewer intersections \(\lightning\)

\(\Rightarrow\) there must exist an edge \(e\) with less than \(2^m\) intersections.
\end{proof}

\begin{corollary}
    String graph recognition is in \(\mathbf{NEXP}\).
\end{corollary}

\begin{proof}
Let \(G\) be a string graph with \(m\) edges.
Then, from \(c_s(m) \leq 4 c_w(2m) + 2m\), we can generate a graph \(G'\) of size \(O(m)\) and \(R\) such that
\((G, R)\) is weakly realizable.

Using Theorem \ref{thm:weak-realizability-bound}, there must be a drawing with \(\leq 2^{O(m)}\) intersections.
\(\leadsto\) The collection of curves also has \(M = 2^{O(m)}\) intersections.

Consider this collection of curves of size \(M\) and draw them as a planar graph (intersection \(\mapsto\) vertex), 
at most \(M\) vertices.

By a result of Schnyder there is a drawing of this graph on an \(M-2 \times M-2\) grid [CLICK to reveal slides].

\(\Rightarrow\) in \(\mathbf{NEXP}\), we can guess a graph on such a grid and verify whether its intersection graph is isomorphic to \(G\).
\end{proof}

\begin{theorem}
    \(c_w(m) \geq 2^{c m}\) for some constant \(c > 0\).
\end{theorem}

\begin{proof}
We start with the planar graph depicted in the following figure:

\image{images/figure-11.png}

The horizontal path \(AB\) is formed by 
the vertices \(A u_n u_{n-1} v_n u_{n-2} v_{n-1} u_{n-3} \cdots u_2 v_3 u_1 v_2 u_0 v_1 B \).

The short vertical edges join the \(u_i\)'s with the bottom path
and no two edges are allowed to cross.

This graph has a topologically unique noncrossing drawing in the plane.
Then we add edges \(\set{u_i, v_i}, i=1,2,\ldots, n \) and allow them to cross
only the edge \(\set{a,b}\) and the horizontal path \(AB\), not each other.

It can be seen that in every weak realization, the edge \(\set{u_i, v_i}\) intersects 
the edge \(\set{a,b}\) at least \(2^{i-1}\) times.

The weak realization of minimum size must look like this:

\image{images/figure-12.png}

Proof by induction:
\(i=1\): \(2^{1-0} = 1\) exactly one crossing of 
\(\set{a,b}\) \(\Rightarrow\) True

Let it hold for some \(i\). The edge \(\set{u_{i+1}, v_{i+1}}\)
is not allowed to cross \(\set{u_i, v_i}\) \(\Rightarrow\) must make 
a turn around \(v_i\) [example \(v_2\)].
Thus, it consists of two parts: one passing from \(u_{i+1}\) to 
the turnpoint, the other from the turnpoint to \(v_{i+1}\).

Both parts may be used as an edge joining \(u_i\) and \(v_i\) 
after contractions of \(\set{u_{i+1}, u_i}\) and \(\set{u_i, v_{i+1}}\) 
into \(u_i\) and a contraction of the turn around \(v_i\) into \(v_i\).

\(\Rightarrow\) (induction hypothesis) each of the two parts share at least 
\(2^{i-1}\) common intersecting points with \(\set{a,b}\)

\(\Rightarrow\) the edge \(\set{u_{i+1}, v_{i+1}}\) shares at least \(2^i\) of them.

The reconstructed graph has \(5n + 13\) edges. We have \(c_w(5n + 13) \geq 2^n - 1\).
\end{proof}

\begin{theorem}
    \(c_s(n) \geq 2^{\hat{c} n}\) for some \(c > 0\).
\end{theorem}

\begin{proof}
Since \(c_s\) and \(c_w\) are polynomially equivalent 
and \(c_w(n) \geq 2^{c n}\), we can find a 
\(\hat{c} > 0\) s.t. \(c_s(n) \geq 2^{\hat{c} n}\).
\end{proof}
\end{document}
